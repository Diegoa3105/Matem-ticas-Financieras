% Options for packages loaded elsewhere
% Options for packages loaded elsewhere
\PassOptionsToPackage{unicode}{hyperref}
\PassOptionsToPackage{hyphens}{url}
\PassOptionsToPackage{dvipsnames,svgnames,x11names}{xcolor}
%
\documentclass[
  letterpaper,
  DIV=11,
  numbers=noendperiod]{scrartcl}
\usepackage{xcolor}
\usepackage{amsmath,amssymb}
\setcounter{secnumdepth}{-\maxdimen} % remove section numbering
\usepackage{iftex}
\ifPDFTeX
  \usepackage[T1]{fontenc}
  \usepackage[utf8]{inputenc}
  \usepackage{textcomp} % provide euro and other symbols
\else % if luatex or xetex
  \usepackage{unicode-math} % this also loads fontspec
  \defaultfontfeatures{Scale=MatchLowercase}
  \defaultfontfeatures[\rmfamily]{Ligatures=TeX,Scale=1}
\fi
\usepackage{lmodern}
\ifPDFTeX\else
  % xetex/luatex font selection
\fi
% Use upquote if available, for straight quotes in verbatim environments
\IfFileExists{upquote.sty}{\usepackage{upquote}}{}
\IfFileExists{microtype.sty}{% use microtype if available
  \usepackage[]{microtype}
  \UseMicrotypeSet[protrusion]{basicmath} % disable protrusion for tt fonts
}{}
\makeatletter
\@ifundefined{KOMAClassName}{% if non-KOMA class
  \IfFileExists{parskip.sty}{%
    \usepackage{parskip}
  }{% else
    \setlength{\parindent}{0pt}
    \setlength{\parskip}{6pt plus 2pt minus 1pt}}
}{% if KOMA class
  \KOMAoptions{parskip=half}}
\makeatother
% Make \paragraph and \subparagraph free-standing
\makeatletter
\ifx\paragraph\undefined\else
  \let\oldparagraph\paragraph
  \renewcommand{\paragraph}{
    \@ifstar
      \xxxParagraphStar
      \xxxParagraphNoStar
  }
  \newcommand{\xxxParagraphStar}[1]{\oldparagraph*{#1}\mbox{}}
  \newcommand{\xxxParagraphNoStar}[1]{\oldparagraph{#1}\mbox{}}
\fi
\ifx\subparagraph\undefined\else
  \let\oldsubparagraph\subparagraph
  \renewcommand{\subparagraph}{
    \@ifstar
      \xxxSubParagraphStar
      \xxxSubParagraphNoStar
  }
  \newcommand{\xxxSubParagraphStar}[1]{\oldsubparagraph*{#1}\mbox{}}
  \newcommand{\xxxSubParagraphNoStar}[1]{\oldsubparagraph{#1}\mbox{}}
\fi
\makeatother

\usepackage{color}
\usepackage{fancyvrb}
\newcommand{\VerbBar}{|}
\newcommand{\VERB}{\Verb[commandchars=\\\{\}]}
\DefineVerbatimEnvironment{Highlighting}{Verbatim}{commandchars=\\\{\}}
% Add ',fontsize=\small' for more characters per line
\usepackage{framed}
\definecolor{shadecolor}{RGB}{241,243,245}
\newenvironment{Shaded}{\begin{snugshade}}{\end{snugshade}}
\newcommand{\AlertTok}[1]{\textcolor[rgb]{0.68,0.00,0.00}{#1}}
\newcommand{\AnnotationTok}[1]{\textcolor[rgb]{0.37,0.37,0.37}{#1}}
\newcommand{\AttributeTok}[1]{\textcolor[rgb]{0.40,0.45,0.13}{#1}}
\newcommand{\BaseNTok}[1]{\textcolor[rgb]{0.68,0.00,0.00}{#1}}
\newcommand{\BuiltInTok}[1]{\textcolor[rgb]{0.00,0.23,0.31}{#1}}
\newcommand{\CharTok}[1]{\textcolor[rgb]{0.13,0.47,0.30}{#1}}
\newcommand{\CommentTok}[1]{\textcolor[rgb]{0.37,0.37,0.37}{#1}}
\newcommand{\CommentVarTok}[1]{\textcolor[rgb]{0.37,0.37,0.37}{\textit{#1}}}
\newcommand{\ConstantTok}[1]{\textcolor[rgb]{0.56,0.35,0.01}{#1}}
\newcommand{\ControlFlowTok}[1]{\textcolor[rgb]{0.00,0.23,0.31}{\textbf{#1}}}
\newcommand{\DataTypeTok}[1]{\textcolor[rgb]{0.68,0.00,0.00}{#1}}
\newcommand{\DecValTok}[1]{\textcolor[rgb]{0.68,0.00,0.00}{#1}}
\newcommand{\DocumentationTok}[1]{\textcolor[rgb]{0.37,0.37,0.37}{\textit{#1}}}
\newcommand{\ErrorTok}[1]{\textcolor[rgb]{0.68,0.00,0.00}{#1}}
\newcommand{\ExtensionTok}[1]{\textcolor[rgb]{0.00,0.23,0.31}{#1}}
\newcommand{\FloatTok}[1]{\textcolor[rgb]{0.68,0.00,0.00}{#1}}
\newcommand{\FunctionTok}[1]{\textcolor[rgb]{0.28,0.35,0.67}{#1}}
\newcommand{\ImportTok}[1]{\textcolor[rgb]{0.00,0.46,0.62}{#1}}
\newcommand{\InformationTok}[1]{\textcolor[rgb]{0.37,0.37,0.37}{#1}}
\newcommand{\KeywordTok}[1]{\textcolor[rgb]{0.00,0.23,0.31}{\textbf{#1}}}
\newcommand{\NormalTok}[1]{\textcolor[rgb]{0.00,0.23,0.31}{#1}}
\newcommand{\OperatorTok}[1]{\textcolor[rgb]{0.37,0.37,0.37}{#1}}
\newcommand{\OtherTok}[1]{\textcolor[rgb]{0.00,0.23,0.31}{#1}}
\newcommand{\PreprocessorTok}[1]{\textcolor[rgb]{0.68,0.00,0.00}{#1}}
\newcommand{\RegionMarkerTok}[1]{\textcolor[rgb]{0.00,0.23,0.31}{#1}}
\newcommand{\SpecialCharTok}[1]{\textcolor[rgb]{0.37,0.37,0.37}{#1}}
\newcommand{\SpecialStringTok}[1]{\textcolor[rgb]{0.13,0.47,0.30}{#1}}
\newcommand{\StringTok}[1]{\textcolor[rgb]{0.13,0.47,0.30}{#1}}
\newcommand{\VariableTok}[1]{\textcolor[rgb]{0.07,0.07,0.07}{#1}}
\newcommand{\VerbatimStringTok}[1]{\textcolor[rgb]{0.13,0.47,0.30}{#1}}
\newcommand{\WarningTok}[1]{\textcolor[rgb]{0.37,0.37,0.37}{\textit{#1}}}

\usepackage{longtable,booktabs,array}
\usepackage{calc} % for calculating minipage widths
% Correct order of tables after \paragraph or \subparagraph
\usepackage{etoolbox}
\makeatletter
\patchcmd\longtable{\par}{\if@noskipsec\mbox{}\fi\par}{}{}
\makeatother
% Allow footnotes in longtable head/foot
\IfFileExists{footnotehyper.sty}{\usepackage{footnotehyper}}{\usepackage{footnote}}
\makesavenoteenv{longtable}
\usepackage{graphicx}
\makeatletter
\newsavebox\pandoc@box
\newcommand*\pandocbounded[1]{% scales image to fit in text height/width
  \sbox\pandoc@box{#1}%
  \Gscale@div\@tempa{\textheight}{\dimexpr\ht\pandoc@box+\dp\pandoc@box\relax}%
  \Gscale@div\@tempb{\linewidth}{\wd\pandoc@box}%
  \ifdim\@tempb\p@<\@tempa\p@\let\@tempa\@tempb\fi% select the smaller of both
  \ifdim\@tempa\p@<\p@\scalebox{\@tempa}{\usebox\pandoc@box}%
  \else\usebox{\pandoc@box}%
  \fi%
}
% Set default figure placement to htbp
\def\fps@figure{htbp}
\makeatother





\setlength{\emergencystretch}{3em} % prevent overfull lines

\providecommand{\tightlist}{%
  \setlength{\itemsep}{0pt}\setlength{\parskip}{0pt}}



 


\KOMAoption{captions}{tableheading}
\makeatletter
\@ifpackageloaded{caption}{}{\usepackage{caption}}
\AtBeginDocument{%
\ifdefined\contentsname
  \renewcommand*\contentsname{Table of contents}
\else
  \newcommand\contentsname{Table of contents}
\fi
\ifdefined\listfigurename
  \renewcommand*\listfigurename{List of Figures}
\else
  \newcommand\listfigurename{List of Figures}
\fi
\ifdefined\listtablename
  \renewcommand*\listtablename{List of Tables}
\else
  \newcommand\listtablename{List of Tables}
\fi
\ifdefined\figurename
  \renewcommand*\figurename{Figure}
\else
  \newcommand\figurename{Figure}
\fi
\ifdefined\tablename
  \renewcommand*\tablename{Table}
\else
  \newcommand\tablename{Table}
\fi
}
\@ifpackageloaded{float}{}{\usepackage{float}}
\floatstyle{ruled}
\@ifundefined{c@chapter}{\newfloat{codelisting}{h}{lop}}{\newfloat{codelisting}{h}{lop}[chapter]}
\floatname{codelisting}{Listing}
\newcommand*\listoflistings{\listof{codelisting}{List of Listings}}
\makeatother
\makeatletter
\makeatother
\makeatletter
\@ifpackageloaded{caption}{}{\usepackage{caption}}
\@ifpackageloaded{subcaption}{}{\usepackage{subcaption}}
\makeatother
\usepackage{bookmark}
\IfFileExists{xurl.sty}{\usepackage{xurl}}{} % add URL line breaks if available
\urlstyle{same}
\hypersetup{
  pdftitle={Funciones anualidades convencionales},
  pdfauthor={Diego Alberto Montes Ramírez, @umich.mx (UMSNH)},
  colorlinks=true,
  linkcolor={blue},
  filecolor={Maroon},
  citecolor={Blue},
  urlcolor={Blue},
  pdfcreator={LaTeX via pandoc}}


\title{Funciones anualidades convencionales}
\author{Diego Alberto Montes Ramírez, @umich.mx (UMSNH)}
\date{}
\begin{document}
\maketitle

\renewcommand*\contentsname{Table of contents}
{
\hypersetup{linkcolor=}
\setcounter{tocdepth}{4}
\tableofcontents
}

\begin{center}
\includegraphics[width=1.5625in,height=\textheight,keepaspectratio]{umich.png}
\end{center}

\section{Anualidades Vencidas (VF)}\label{anualidades-vencidas-vf}

Lo primero que haremos será crear las diferentes funciones para las
anualidades vencidas con valor futuro:

\begin{enumerate}
\def\labelenumi{\arabic{enumi}.}
\tightlist
\item
  Valor futuro (VF)
\end{enumerate}

\begin{Shaded}
\begin{Highlighting}[]
\NormalTok{AnualidadesVencidasVFVF }\OtherTok{=} \ControlFlowTok{function}\NormalTok{(A, r, T)\{}
\NormalTok{   VF }\OtherTok{=}\NormalTok{ A }\SpecialCharTok{*}\NormalTok{ ( ( (}\DecValTok{1} \SpecialCharTok{+}\NormalTok{ r)}\SpecialCharTok{\^{}}\NormalTok{T }\SpecialCharTok{{-}} \DecValTok{1}\NormalTok{ ) }\SpecialCharTok{/}\NormalTok{ r )}
   \FunctionTok{return}\NormalTok{(VF)}
\NormalTok{\}}
\end{Highlighting}
\end{Shaded}

\begin{enumerate}
\def\labelenumi{\arabic{enumi}.}
\setcounter{enumi}{1}
\tightlist
\item
  Monto del pago en la anualidad (A)
\end{enumerate}

\begin{Shaded}
\begin{Highlighting}[]
\NormalTok{AnualidadesVencidasVFA }\OtherTok{=} \ControlFlowTok{function}\NormalTok{(VF, r, T)\{}
\NormalTok{  A }\OtherTok{=}\NormalTok{ VF }\SpecialCharTok{*}\NormalTok{ ( r }\SpecialCharTok{/}\NormalTok{ ( (}\DecValTok{1} \SpecialCharTok{+}\NormalTok{ r)}\SpecialCharTok{\^{}}\NormalTok{T }\SpecialCharTok{{-}} \DecValTok{1}\NormalTok{ ) )}
  \FunctionTok{return}\NormalTok{(A)}
\NormalTok{\}}
\end{Highlighting}
\end{Shaded}

\begin{enumerate}
\def\labelenumi{\arabic{enumi}.}
\setcounter{enumi}{2}
\tightlist
\item
  Tasa de interes del periodo (r)
\end{enumerate}

\begin{Shaded}
\begin{Highlighting}[]
\NormalTok{AnualidadesVencidasVFr }\OtherTok{=} \ControlFlowTok{function}\NormalTok{(VF, A, T, }\AttributeTok{guess =} \FloatTok{0.1}\NormalTok{)\{}

\NormalTok{  f }\OtherTok{\textless{}{-}} \ControlFlowTok{function}\NormalTok{(r)\{}
\NormalTok{    A }\SpecialCharTok{*}\NormalTok{ (((}\DecValTok{1}\SpecialCharTok{+}\NormalTok{r)}\SpecialCharTok{\^{}}\NormalTok{T }\SpecialCharTok{{-}} \DecValTok{1}\NormalTok{) }\SpecialCharTok{/}\NormalTok{ r) }\SpecialCharTok{{-}}\NormalTok{ VF}
\NormalTok{  \}}

\NormalTok{  sol }\OtherTok{\textless{}{-}} \FunctionTok{uniroot}\NormalTok{(f, }\AttributeTok{interval =} \FunctionTok{c}\NormalTok{(}\FloatTok{1e{-}8}\NormalTok{, }\DecValTok{1}\NormalTok{), }\AttributeTok{extendInt =} \StringTok{"yes"}\NormalTok{)}

  \FunctionTok{return}\NormalTok{(sol}\SpecialCharTok{$}\NormalTok{root)}
\NormalTok{\}}
\end{Highlighting}
\end{Shaded}

\begin{enumerate}
\def\labelenumi{\arabic{enumi}.}
\setcounter{enumi}{3}
\tightlist
\item
  Número de periodos de pagos (T)
\end{enumerate}

\begin{Shaded}
\begin{Highlighting}[]
\NormalTok{AnualidadesVencidasVFT }\OtherTok{=} \ControlFlowTok{function}\NormalTok{(VF, A, r)\{}
\NormalTok{  T }\OtherTok{=} \FunctionTok{log}\NormalTok{(}\DecValTok{1} \SpecialCharTok{+}\NormalTok{ (VF }\SpecialCharTok{*}\NormalTok{ r }\SpecialCharTok{/}\NormalTok{ A)) }\SpecialCharTok{/} \FunctionTok{log}\NormalTok{(}\DecValTok{1} \SpecialCharTok{+}\NormalTok{ r)}
  \FunctionTok{return}\NormalTok{(T)}
\NormalTok{\}}
\end{Highlighting}
\end{Shaded}

\section{Anualidades Venciadas (VA)}\label{anualidades-venciadas-va}

Ahora lo que haremos será definir las funciones para anualidades
vencidas con valor actual:

\begin{enumerate}
\def\labelenumi{\arabic{enumi}.}
\tightlist
\item
  Valor Actual (VA)
\end{enumerate}

\begin{Shaded}
\begin{Highlighting}[]
\NormalTok{AnualidadesVencidasVAVA }\OtherTok{=} \ControlFlowTok{function}\NormalTok{(A, r, T)\{}
\NormalTok{  VA }\OtherTok{=}\NormalTok{  A }\SpecialCharTok{*}\NormalTok{ (}\DecValTok{1} \SpecialCharTok{{-}}\NormalTok{ (}\DecValTok{1} \SpecialCharTok{+}\NormalTok{ r)}\SpecialCharTok{\^{}}\NormalTok{(}\SpecialCharTok{{-}}\NormalTok{T)) }\SpecialCharTok{/}\NormalTok{ r}
  \FunctionTok{return}\NormalTok{(VA)}
\NormalTok{\}}
\end{Highlighting}
\end{Shaded}

\begin{enumerate}
\def\labelenumi{\arabic{enumi}.}
\setcounter{enumi}{1}
\tightlist
\item
  Monto del pago en la anualidad (A)
\end{enumerate}

\begin{Shaded}
\begin{Highlighting}[]
\NormalTok{AnualidadesVencidasVAA }\OtherTok{=} \ControlFlowTok{function}\NormalTok{(VA, r, T)\{}
\NormalTok{  A }\OtherTok{=}\NormalTok{ VA }\SpecialCharTok{*}\NormalTok{ r }\SpecialCharTok{/}\NormalTok{ (}\DecValTok{1} \SpecialCharTok{{-}}\NormalTok{ (}\DecValTok{1} \SpecialCharTok{+}\NormalTok{ r)}\SpecialCharTok{\^{}}\NormalTok{(}\SpecialCharTok{{-}}\NormalTok{T))}
  \FunctionTok{return}\NormalTok{(A)}
\NormalTok{\}}
\end{Highlighting}
\end{Shaded}

\begin{enumerate}
\def\labelenumi{\arabic{enumi}.}
\setcounter{enumi}{2}
\tightlist
\item
  Tasa de interes del periodo (r)
\end{enumerate}

\begin{Shaded}
\begin{Highlighting}[]
\NormalTok{AnualidadesVencidasVAr }\OtherTok{=} \ControlFlowTok{function}\NormalTok{(VA, A, T, }\AttributeTok{guess =} \FloatTok{0.1}\NormalTok{)\{}

\NormalTok{  f }\OtherTok{\textless{}{-}} \ControlFlowTok{function}\NormalTok{(r)\{}
\NormalTok{    A }\SpecialCharTok{*}\NormalTok{ (}\DecValTok{1} \SpecialCharTok{{-}}\NormalTok{ (}\DecValTok{1}\SpecialCharTok{+}\NormalTok{r)}\SpecialCharTok{\^{}}\NormalTok{(}\SpecialCharTok{{-}}\NormalTok{T)) }\SpecialCharTok{/}\NormalTok{ r }\SpecialCharTok{{-}}\NormalTok{ VA}
\NormalTok{  \}}

\NormalTok{  sol }\OtherTok{\textless{}{-}} \FunctionTok{uniroot}\NormalTok{(f, }\AttributeTok{interval =} \FunctionTok{c}\NormalTok{(}\FloatTok{1e{-}8}\NormalTok{, }\DecValTok{1}\NormalTok{), }\AttributeTok{extendInt =} \StringTok{"yes"}\NormalTok{)}

  \FunctionTok{return}\NormalTok{(sol}\SpecialCharTok{$}\NormalTok{root)}
\NormalTok{\}}
\end{Highlighting}
\end{Shaded}

\begin{enumerate}
\def\labelenumi{\arabic{enumi}.}
\setcounter{enumi}{3}
\tightlist
\item
  Número de periodos de pagos (T)
\end{enumerate}

\begin{Shaded}
\begin{Highlighting}[]
\NormalTok{AnualidadesVencidasVAT }\OtherTok{=} \ControlFlowTok{function}\NormalTok{(VA, A, r)\{}
\NormalTok{  T }\OtherTok{=} \SpecialCharTok{{-}}\FunctionTok{log}\NormalTok{(}\DecValTok{1} \SpecialCharTok{{-}}\NormalTok{ (VA }\SpecialCharTok{*}\NormalTok{ r }\SpecialCharTok{/}\NormalTok{ A)) }\SpecialCharTok{/} \FunctionTok{log}\NormalTok{(}\DecValTok{1} \SpecialCharTok{+}\NormalTok{ r)}
  \FunctionTok{return}\NormalTok{(T)}
\NormalTok{\}}
\end{Highlighting}
\end{Shaded}

\section{Anualidades Anticipadas (VF)}\label{anualidades-anticipadas-vf}

Lo siguiente que haremos será definir las funciones de Anualidades
anticipadas con Valor futuro:

\begin{enumerate}
\def\labelenumi{\arabic{enumi}.}
\tightlist
\item
  Valor futuro (VF)
\end{enumerate}

\begin{Shaded}
\begin{Highlighting}[]
\NormalTok{AnualidadesAnticipadasVFVF }\OtherTok{=} \ControlFlowTok{function}\NormalTok{(A, r, T)\{}
\NormalTok{  VF }\OtherTok{=}\NormalTok{  A }\SpecialCharTok{*}\NormalTok{ (((}\DecValTok{1} \SpecialCharTok{+}\NormalTok{ r)}\SpecialCharTok{\^{}}\NormalTok{T }\SpecialCharTok{{-}} \DecValTok{1}\NormalTok{) }\SpecialCharTok{/}\NormalTok{ r) }\SpecialCharTok{*}\NormalTok{ (}\DecValTok{1} \SpecialCharTok{+}\NormalTok{ r)}
  \FunctionTok{return}\NormalTok{(VF)}
\NormalTok{\}}
\end{Highlighting}
\end{Shaded}

\begin{enumerate}
\def\labelenumi{\arabic{enumi}.}
\setcounter{enumi}{1}
\tightlist
\item
  Monto del pago en la anualidad (A)
\end{enumerate}

\begin{Shaded}
\begin{Highlighting}[]
\NormalTok{AnualidadesAnticipadasVFA }\OtherTok{=} \ControlFlowTok{function}\NormalTok{(VF, r, T)\{}
\NormalTok{  A }\OtherTok{=}\NormalTok{ VF }\SpecialCharTok{*}\NormalTok{ r }\SpecialCharTok{/}\NormalTok{ (((}\DecValTok{1} \SpecialCharTok{+}\NormalTok{ r)}\SpecialCharTok{\^{}}\NormalTok{T }\SpecialCharTok{{-}} \DecValTok{1}\NormalTok{) }\SpecialCharTok{*}\NormalTok{ (}\DecValTok{1} \SpecialCharTok{+}\NormalTok{ r))}
  \FunctionTok{return}\NormalTok{(A)}
\NormalTok{\}}
\end{Highlighting}
\end{Shaded}

\begin{enumerate}
\def\labelenumi{\arabic{enumi}.}
\setcounter{enumi}{2}
\tightlist
\item
  Tasa de interes del periodo (r)
\end{enumerate}

\begin{Shaded}
\begin{Highlighting}[]
\NormalTok{AnualidadesAnticipadasVFr }\OtherTok{=} \ControlFlowTok{function}\NormalTok{(VF, A, T, }\AttributeTok{guess =} \FloatTok{0.1}\NormalTok{)\{}

\NormalTok{  f }\OtherTok{\textless{}{-}} \ControlFlowTok{function}\NormalTok{(r)\{}
\NormalTok{    A }\SpecialCharTok{*}\NormalTok{ (((}\DecValTok{1}\SpecialCharTok{+}\NormalTok{r)}\SpecialCharTok{\^{}}\NormalTok{T }\SpecialCharTok{{-}} \DecValTok{1}\NormalTok{) }\SpecialCharTok{/}\NormalTok{ r) }\SpecialCharTok{*}\NormalTok{ (}\DecValTok{1}\SpecialCharTok{+}\NormalTok{r) }\SpecialCharTok{{-}}\NormalTok{ VF}
\NormalTok{  \}}

\NormalTok{  sol }\OtherTok{\textless{}{-}} \FunctionTok{uniroot}\NormalTok{(f, }\AttributeTok{interval =} \FunctionTok{c}\NormalTok{(}\FloatTok{1e{-}8}\NormalTok{, }\DecValTok{1}\NormalTok{), }\AttributeTok{extendInt =} \StringTok{"yes"}\NormalTok{)}

  \FunctionTok{return}\NormalTok{(sol}\SpecialCharTok{$}\NormalTok{root)}
\NormalTok{\}}
\end{Highlighting}
\end{Shaded}

\begin{enumerate}
\def\labelenumi{\arabic{enumi}.}
\setcounter{enumi}{3}
\tightlist
\item
  Número de periodos de pago (T)
\end{enumerate}

\begin{Shaded}
\begin{Highlighting}[]
\NormalTok{AnualidadesAnticipadasVFT }\OtherTok{=} \ControlFlowTok{function}\NormalTok{(VF, A, r)\{}
\NormalTok{  T }\OtherTok{=} \FunctionTok{log}\NormalTok{(}\DecValTok{1} \SpecialCharTok{+}\NormalTok{ (VF }\SpecialCharTok{*}\NormalTok{ r }\SpecialCharTok{/}\NormalTok{ (A }\SpecialCharTok{*}\NormalTok{ (}\DecValTok{1} \SpecialCharTok{+}\NormalTok{ r)))) }\SpecialCharTok{/} \FunctionTok{log}\NormalTok{(}\DecValTok{1} \SpecialCharTok{+}\NormalTok{ r)}
  \FunctionTok{return}\NormalTok{(T)}
\NormalTok{\}}
\end{Highlighting}
\end{Shaded}

\section{Anualidades Anticipadas (VA)}\label{anualidades-anticipadas-va}

Casi por terminar, vamos a crear las funciones para cada parte de
anualidades anticipadas con valor actual:

\begin{enumerate}
\def\labelenumi{\arabic{enumi}.}
\tightlist
\item
  Valor actual (VA)
\end{enumerate}

\begin{Shaded}
\begin{Highlighting}[]
\NormalTok{AnualidadesAnticipadasVAVA }\OtherTok{=}\ControlFlowTok{function}\NormalTok{(A, r, T)\{}
\NormalTok{  VA }\OtherTok{=}\NormalTok{ A }\SpecialCharTok{*}\NormalTok{ (}\DecValTok{1} \SpecialCharTok{{-}}\NormalTok{ (}\DecValTok{1} \SpecialCharTok{+}\NormalTok{ r)}\SpecialCharTok{\^{}}\NormalTok{(}\SpecialCharTok{{-}}\NormalTok{T)) }\SpecialCharTok{/}\NormalTok{ r }\SpecialCharTok{*}\NormalTok{ (}\DecValTok{1} \SpecialCharTok{+}\NormalTok{ r)}
  \FunctionTok{return}\NormalTok{(VA)}
\NormalTok{\}}
\end{Highlighting}
\end{Shaded}

\begin{enumerate}
\def\labelenumi{\arabic{enumi}.}
\setcounter{enumi}{1}
\tightlist
\item
  Monto del pago en la anualidad (A)
\end{enumerate}

\begin{Shaded}
\begin{Highlighting}[]
\NormalTok{AnualidadesAnticipadasVAA }\OtherTok{=} \ControlFlowTok{function}\NormalTok{(VA, r, T)\{}
\NormalTok{  A }\OtherTok{=}\NormalTok{  VA }\SpecialCharTok{*}\NormalTok{ r }\SpecialCharTok{/}\NormalTok{ ((}\DecValTok{1} \SpecialCharTok{{-}}\NormalTok{ (}\DecValTok{1} \SpecialCharTok{+}\NormalTok{ r)}\SpecialCharTok{\^{}}\NormalTok{(}\SpecialCharTok{{-}}\NormalTok{T)) }\SpecialCharTok{*}\NormalTok{ (}\DecValTok{1} \SpecialCharTok{+}\NormalTok{ r))}
  \FunctionTok{return}\NormalTok{ (A)}
\NormalTok{\}}
\end{Highlighting}
\end{Shaded}

\begin{enumerate}
\def\labelenumi{\arabic{enumi}.}
\setcounter{enumi}{2}
\tightlist
\item
  Tasa de interes del periodo (r)
\end{enumerate}

\begin{Shaded}
\begin{Highlighting}[]
\NormalTok{AnualidadesAnticipadasVAr }\OtherTok{=} \ControlFlowTok{function}\NormalTok{(VA, A, T, }\AttributeTok{guess =} \FloatTok{0.1}\NormalTok{)\{}

\NormalTok{  f }\OtherTok{\textless{}{-}} \ControlFlowTok{function}\NormalTok{(r)\{}
\NormalTok{    A }\SpecialCharTok{*}\NormalTok{ (}\DecValTok{1} \SpecialCharTok{{-}}\NormalTok{ (}\DecValTok{1}\SpecialCharTok{+}\NormalTok{r)}\SpecialCharTok{\^{}}\NormalTok{(}\SpecialCharTok{{-}}\NormalTok{T)) }\SpecialCharTok{/}\NormalTok{ r }\SpecialCharTok{*}\NormalTok{ (}\DecValTok{1}\SpecialCharTok{+}\NormalTok{r) }\SpecialCharTok{{-}}\NormalTok{ VA}
\NormalTok{  \}}

\NormalTok{  sol }\OtherTok{\textless{}{-}} \FunctionTok{uniroot}\NormalTok{(f, }\AttributeTok{interval =} \FunctionTok{c}\NormalTok{(}\FloatTok{1e{-}8}\NormalTok{, }\DecValTok{1}\NormalTok{), }\AttributeTok{extendInt =} \StringTok{"yes"}\NormalTok{)}

  \FunctionTok{return}\NormalTok{(sol}\SpecialCharTok{$}\NormalTok{root)}
\NormalTok{\}}
\end{Highlighting}
\end{Shaded}

\begin{enumerate}
\def\labelenumi{\arabic{enumi}.}
\setcounter{enumi}{3}
\tightlist
\item
  Número de periodos de pago (T)
\end{enumerate}

\begin{Shaded}
\begin{Highlighting}[]
\NormalTok{AnualidadesAnticipadasVAT }\OtherTok{=} \ControlFlowTok{function}\NormalTok{(VA, A, r)\{}
\NormalTok{  T }\OtherTok{=} \SpecialCharTok{{-}}\FunctionTok{log}\NormalTok{(}\DecValTok{1} \SpecialCharTok{{-}}\NormalTok{ (VA }\SpecialCharTok{*}\NormalTok{ r }\SpecialCharTok{/}\NormalTok{ (A }\SpecialCharTok{*}\NormalTok{ (}\DecValTok{1} \SpecialCharTok{+}\NormalTok{ r)))) }\SpecialCharTok{/} \FunctionTok{log}\NormalTok{(}\DecValTok{1} \SpecialCharTok{+}\NormalTok{ r)}
  \FunctionTok{return}\NormalTok{(T)}
\NormalTok{\}}
\end{Highlighting}
\end{Shaded}

\section{Funciones Generales}\label{funciones-generales}

Por último, necesitamos incluir las ultimas dos funciones generales
¨valorActualAnualidades¨ y ¨valorFuturoAnualidades¨ para que esas dos
funciones puedan llamar a las otras dependeindo el valor que nos pidan
encontrar:

\begin{enumerate}
\def\labelenumi{\arabic{enumi}.}
\tightlist
\item
  valorFuturoAnualidades
\end{enumerate}

\begin{Shaded}
\begin{Highlighting}[]
\NormalTok{valorFuturoAnualidades }\OtherTok{\textless{}{-}} \ControlFlowTok{function}\NormalTok{(}\AttributeTok{A =} \ConstantTok{NA}\NormalTok{, }\AttributeTok{r =} \ConstantTok{NA}\NormalTok{, }\AttributeTok{T =} \ConstantTok{NA}\NormalTok{, }\AttributeTok{VF =} \ConstantTok{NA}\NormalTok{, }\AttributeTok{anticipada =} \ConstantTok{FALSE}\NormalTok{) \{}

  \ControlFlowTok{if}\NormalTok{ (}\SpecialCharTok{!}\FunctionTok{is.na}\NormalTok{(r) }\SpecialCharTok{\&\&}\NormalTok{ r }\SpecialCharTok{\textless{}} \DecValTok{0}\NormalTok{) }\FunctionTok{stop}\NormalTok{(}\StringTok{"Error: la tasa de interés (r) no puede ser negativa."}\NormalTok{)}
  \ControlFlowTok{if}\NormalTok{ (}\SpecialCharTok{!}\FunctionTok{is.na}\NormalTok{(T) }\SpecialCharTok{\&\&}\NormalTok{ T }\SpecialCharTok{\textless{}=} \DecValTok{0}\NormalTok{) }\FunctionTok{stop}\NormalTok{(}\StringTok{"Error: el número de periodos (T) debe ser mayor que cero."}\NormalTok{)}
  \ControlFlowTok{if}\NormalTok{ (}\SpecialCharTok{!}\FunctionTok{is.logical}\NormalTok{(anticipada)) }\FunctionTok{stop}\NormalTok{(}\StringTok{"Error: \textquotesingle{}anticipada\textquotesingle{} debe ser TRUE o FALSE."}\NormalTok{)}

 
\NormalTok{  cantidad\_NA }\OtherTok{\textless{}{-}} \FunctionTok{sum}\NormalTok{(}\FunctionTok{is.na}\NormalTok{(}\FunctionTok{c}\NormalTok{(A, r, T, VF)))}

  \ControlFlowTok{if}\NormalTok{ (cantidad\_NA }\SpecialCharTok{!=} \DecValTok{1}\NormalTok{) \{}
    \FunctionTok{stop}\NormalTok{(}\StringTok{"Error: Debes dejar exactamente uno de los valores como NA (A, r, T o VF)."}\NormalTok{)}
\NormalTok{  \}}

  
  \ControlFlowTok{if}\NormalTok{ (}\FunctionTok{is.na}\NormalTok{(VF)) \{}
    \ControlFlowTok{if}\NormalTok{ (anticipada) \{}
\NormalTok{      VF }\OtherTok{\textless{}{-}} \FunctionTok{AnualidadesAnticipadasVFVF}\NormalTok{(A, r, T)}
\NormalTok{    \} }\ControlFlowTok{else}\NormalTok{ \{}
\NormalTok{      VF }\OtherTok{\textless{}{-}} \FunctionTok{AnualidadesVencidasVFVF}\NormalTok{(A, r, T)}
\NormalTok{    \}}
    \FunctionTok{return}\NormalTok{(VF)}
\NormalTok{  \}}

  
  \ControlFlowTok{if}\NormalTok{ (}\FunctionTok{is.na}\NormalTok{(A)) \{}
    \ControlFlowTok{if}\NormalTok{ (anticipada) \{}
\NormalTok{      A }\OtherTok{\textless{}{-}} \FunctionTok{AnualidadesAnticipadasVFA}\NormalTok{(VF, r, T)}
\NormalTok{    \} }\ControlFlowTok{else}\NormalTok{ \{}
\NormalTok{      A }\OtherTok{\textless{}{-}} \FunctionTok{AnualidadesVencidasVFA}\NormalTok{(VF, r, T)}
\NormalTok{    \}}
    \FunctionTok{return}\NormalTok{(A)}
\NormalTok{  \}}

  
  \ControlFlowTok{if}\NormalTok{ (}\FunctionTok{is.na}\NormalTok{(r)) \{}
    \ControlFlowTok{if}\NormalTok{ (anticipada) \{}
\NormalTok{      r }\OtherTok{\textless{}{-}} \FunctionTok{AnualidadesAnticipadasVFr}\NormalTok{(VF, A, T)}
\NormalTok{    \} }\ControlFlowTok{else}\NormalTok{ \{}
\NormalTok{      r }\OtherTok{\textless{}{-}} \FunctionTok{AnualidadesVencidasVFr}\NormalTok{(VF, A, T)}
\NormalTok{    \}}
    \FunctionTok{return}\NormalTok{(r)}
\NormalTok{  \}}

  
  \ControlFlowTok{if}\NormalTok{ (}\FunctionTok{is.na}\NormalTok{(T)) \{}
    \ControlFlowTok{if}\NormalTok{ (anticipada) \{}
\NormalTok{      T }\OtherTok{\textless{}{-}} \FunctionTok{AnualidadesAnticipadasVFT}\NormalTok{(VF, A, r)}
\NormalTok{    \} }\ControlFlowTok{else}\NormalTok{ \{}
\NormalTok{      T }\OtherTok{\textless{}{-}} \FunctionTok{AnualidadesVencidasVFT}\NormalTok{(VF, A, r)}
\NormalTok{    \}}
    \FunctionTok{return}\NormalTok{(T)}
\NormalTok{  \}}

  
  \FunctionTok{stop}\NormalTok{(}\StringTok{"Error inesperado: revisa los valores nuevamente."}\NormalTok{)}
\NormalTok{\}}
\end{Highlighting}
\end{Shaded}

\begin{enumerate}
\def\labelenumi{\arabic{enumi}.}
\setcounter{enumi}{1}
\tightlist
\item
  valorActualAnualidades
\end{enumerate}

\begin{Shaded}
\begin{Highlighting}[]
\NormalTok{valorActualAnualidades }\OtherTok{\textless{}{-}} \ControlFlowTok{function}\NormalTok{(}\AttributeTok{A =} \ConstantTok{NA}\NormalTok{, }\AttributeTok{r =} \ConstantTok{NA}\NormalTok{, }\AttributeTok{T =} \ConstantTok{NA}\NormalTok{, }\AttributeTok{VA =} \ConstantTok{NA}\NormalTok{, }\AttributeTok{anticipada =} \ConstantTok{FALSE}\NormalTok{) \{}

  
  \ControlFlowTok{if}\NormalTok{ (}\SpecialCharTok{!}\FunctionTok{is.na}\NormalTok{(r) }\SpecialCharTok{\&\&}\NormalTok{ r }\SpecialCharTok{\textless{}} \DecValTok{0}\NormalTok{) }\FunctionTok{stop}\NormalTok{(}\StringTok{"Error: la tasa de interés (r) no puede ser negativa."}\NormalTok{)}
  \ControlFlowTok{if}\NormalTok{ (}\SpecialCharTok{!}\FunctionTok{is.na}\NormalTok{(T) }\SpecialCharTok{\&\&}\NormalTok{ T }\SpecialCharTok{\textless{}=} \DecValTok{0}\NormalTok{) }\FunctionTok{stop}\NormalTok{(}\StringTok{"Error: el número de periodos (T) debe ser mayor que cero."}\NormalTok{)}
  \ControlFlowTok{if}\NormalTok{ (}\SpecialCharTok{!}\FunctionTok{is.logical}\NormalTok{(anticipada)) }\FunctionTok{stop}\NormalTok{(}\StringTok{"Error: \textquotesingle{}anticipada\textquotesingle{} debe ser TRUE o FALSE."}\NormalTok{)}

  
\NormalTok{  cantidad\_NA }\OtherTok{\textless{}{-}} \FunctionTok{sum}\NormalTok{(}\FunctionTok{is.na}\NormalTok{(}\FunctionTok{c}\NormalTok{(A, r, T, VA)))}

  \ControlFlowTok{if}\NormalTok{ (cantidad\_NA }\SpecialCharTok{!=} \DecValTok{1}\NormalTok{) \{}
    \FunctionTok{stop}\NormalTok{(}\StringTok{"Error: Debes dejar exactamente un valor como NA (A, r, T o VA)."}\NormalTok{)}
\NormalTok{  \}}

 
  \ControlFlowTok{if}\NormalTok{ (}\FunctionTok{is.na}\NormalTok{(VA)) \{}
    \ControlFlowTok{if}\NormalTok{ (anticipada) \{}
\NormalTok{      VA }\OtherTok{\textless{}{-}} \FunctionTok{AnualidadesAnticipadasVAVA}\NormalTok{(A, r, T)}
\NormalTok{    \} }\ControlFlowTok{else}\NormalTok{ \{}
\NormalTok{      VA }\OtherTok{\textless{}{-}} \FunctionTok{AnualidadesVencidasVAVA}\NormalTok{(A, r, T)}
\NormalTok{    \}}
    \FunctionTok{return}\NormalTok{(VA)}
\NormalTok{  \}}

  
  \ControlFlowTok{if}\NormalTok{ (}\FunctionTok{is.na}\NormalTok{(A)) \{}
    \ControlFlowTok{if}\NormalTok{ (anticipada) \{}
\NormalTok{      A }\OtherTok{\textless{}{-}} \FunctionTok{AnualidadesAnticipadasVAA}\NormalTok{(VA, r, T)}
\NormalTok{    \} }\ControlFlowTok{else}\NormalTok{ \{}
\NormalTok{      A }\OtherTok{\textless{}{-}} \FunctionTok{AnualidadesVencidasVAA}\NormalTok{(VA, r, T)}
\NormalTok{    \}}
    \FunctionTok{return}\NormalTok{(A)}
\NormalTok{  \}}

  
  \ControlFlowTok{if}\NormalTok{ (}\FunctionTok{is.na}\NormalTok{(r)) \{}
    \ControlFlowTok{if}\NormalTok{ (anticipada) \{}
\NormalTok{      r }\OtherTok{\textless{}{-}} \FunctionTok{AnualidadesAnticipadasVAr}\NormalTok{(VA, A, T)}
\NormalTok{    \} }\ControlFlowTok{else}\NormalTok{ \{}
\NormalTok{      r }\OtherTok{\textless{}{-}} \FunctionTok{AnualidadesVencidasVAr}\NormalTok{(VA, A, T)}
\NormalTok{    \}}
    \FunctionTok{return}\NormalTok{(r)}
\NormalTok{  \}}

  
  \ControlFlowTok{if}\NormalTok{ (}\FunctionTok{is.na}\NormalTok{(T)) \{}
    \ControlFlowTok{if}\NormalTok{ (anticipada) \{}
\NormalTok{      T }\OtherTok{\textless{}{-}} \FunctionTok{AnualidadesAnticipadasVAT}\NormalTok{(VA, A, r)}
\NormalTok{    \} }\ControlFlowTok{else}\NormalTok{ \{}
\NormalTok{      T }\OtherTok{\textless{}{-}} \FunctionTok{AnualidadesVencidasVAT}\NormalTok{(VA, A, r)}
\NormalTok{    \}}
    \FunctionTok{return}\NormalTok{(T)}
\NormalTok{  \}}

 
  \FunctionTok{stop}\NormalTok{(}\StringTok{"Error inesperado: revisa los datos ingresados."}\NormalTok{)}
\NormalTok{\}}
\end{Highlighting}
\end{Shaded}

Ejemplos:

\begin{enumerate}
\def\labelenumi{\arabic{enumi}.}
\tightlist
\item
\end{enumerate}

\begin{Shaded}
\begin{Highlighting}[]
\FunctionTok{valorFuturoAnualidades}\NormalTok{(}\AttributeTok{r=}\FloatTok{0.02}\NormalTok{,}\AttributeTok{VF=}\DecValTok{1000000}\NormalTok{,}\AttributeTok{T=}\DecValTok{360}\NormalTok{,}\AttributeTok{anticipada=}\ConstantTok{TRUE}\NormalTok{)}
\end{Highlighting}
\end{Shaded}

\begin{verbatim}
[1] 15.72955
\end{verbatim}

\begin{enumerate}
\def\labelenumi{\arabic{enumi}.}
\setcounter{enumi}{1}
\tightlist
\item
\end{enumerate}

\begin{Shaded}
\begin{Highlighting}[]
\FunctionTok{valorFuturoAnualidades}\NormalTok{(}\AttributeTok{r=}\FloatTok{0.02}\NormalTok{,}\AttributeTok{A=}\DecValTok{1500}\NormalTok{,}\AttributeTok{T=}\DecValTok{360}\NormalTok{,}\AttributeTok{anticipada=}\ConstantTok{TRUE}\NormalTok{)}
\end{Highlighting}
\end{Shaded}

\begin{verbatim}
[1] 95361926
\end{verbatim}

\begin{enumerate}
\def\labelenumi{\arabic{enumi}.}
\setcounter{enumi}{2}
\tightlist
\item
\end{enumerate}

\begin{Shaded}
\begin{Highlighting}[]
\FunctionTok{valorFuturoAnualidades}\NormalTok{(}\AttributeTok{VF=}\DecValTok{30000}\NormalTok{, }\AttributeTok{A=}\DecValTok{400}\NormalTok{, }\AttributeTok{T=}\DecValTok{48}\NormalTok{, }\AttributeTok{anticipada=}\ConstantTok{TRUE}\NormalTok{)}
\end{Highlighting}
\end{Shaded}

\begin{verbatim}
[1] 0.01721927
\end{verbatim}

\begin{enumerate}
\def\labelenumi{\arabic{enumi}.}
\setcounter{enumi}{3}
\tightlist
\item
\end{enumerate}

\begin{Shaded}
\begin{Highlighting}[]
\FunctionTok{valorActualAnualidades}\NormalTok{(}\AttributeTok{anticipada=}\ConstantTok{FALSE}\NormalTok{, }\AttributeTok{VA=}\DecValTok{23000}\NormalTok{, }\AttributeTok{r=}\FloatTok{0.0125}\NormalTok{, }\AttributeTok{T=}\DecValTok{48}\NormalTok{)}
\end{Highlighting}
\end{Shaded}

\begin{verbatim}
[1] 640.1072
\end{verbatim}




\end{document}
